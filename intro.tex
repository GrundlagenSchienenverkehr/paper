\chapter{Einleitung}

% schema: http://www.cs.columbia.edu/~hgs/etc/intro-style.html

% Motivation für das Problemfeld
Die mediale Berichterstattung über Infrastrukturprojekte ist geprägt vom Widerstand der
Bürger gegen diese. Als bisheriger Höhepunkt dessen ist sicherlich die politische
Auseinandersetzung sowie der Medientrubel rund um \emph{Stuttgart 21} im
Jahr 2010 zu nennen. Im Mittelpunkt solcher Konflikte stehen neben den raumgestalterischen
oft auch die finanziellen Aspekte. Wichtig für diese ist vor allem das Verhältnis zwischen
Baukosten und entstehendem Nutzen von Neubaustrecken, der bei einigen
Prestigeprojekten sehr niedrig liegt, oft auch wegen einer viel zu geringen Einschätzung
der Kosten in der Planungsphase.

% spezifisches Problem
Die vorliegenden Ausarbeitung untersucht die Kosten zum Bau von Infrastruktur im
schienengebundenen Stadtverkehr. Die gewonnen Erkenntnisse könnten genutzt werden, um die
Kosten von Stadt- und Straßenbahnprojekten abzuschätzen und zu bewerten.

% herangehensweise, Hauptaussagen
Zur Berechnung eines geeigneten Kostenmaßes werden unterschiedliche
Neubaustrecken verglichen und ausgewertet. Es wird versucht, geeignete
Kennwerte zu ermitteln und mittels linearer Regression die gewonnenen Daten
für relevante Streckenmerkenmale zu generalisieren. Es wird versucht, geeignete Kennwerte
für die unterschiedlichen Streckenklassen zu ermitteln. Hierzu werden die Strecken in Straßenbahnen
und U-Bahnen unterteilt und auf ihre Parameter untersucht. Zum historischen
Vergleich werden zudem Neubauten in Darmstadt aus der Zeit um 1900 hinzugezogen.

% Struktur der restlichen Arbeit
Im weiteren Verlauf der Arbeit werden zunächst der Untersuchungsansatz und die
Datenerhebung genauer erläutert. Im nächsten Abschnitt werden die Rechercheergebnisse
für die verschiedene Strecken untersucht. Anschließend werden diese Daten in Relation
gesetzt sowie ein Preismodell zum Bau von Nahverkehrsinfrastruktur erarbeitet. Im letzten
Abschnitt werden die Ergebnisse zusammengefasst und mögliche Nachteile unserer Methode
aufgezeigt.
