\section{Einleitung}

% schema: http://www.cs.columbia.edu/~hgs/etc/intro-style.html

% Motivation für das Problemfeld
Die mediale Berichterstattung über Infrastrukturprojekte ist gebrägt, vom
Widerstand der Bürger gegen diese Projekte. Als Höhepunkt dessen ist sicherlich
die politische Auseinadersetzung sowie die Medientrubel rund um die Protesste
gegen \emph{Stuttgart 21}. Im Mittelpunkt solcher Auseinadersetzung stehen neben
den raumgestallterischen oft auch die finanziellen Aspekte. Wichtig für diese
Fragestelltung ist vorallem der Quotient zwischen Herrstellungskosten und
Nutzen von Neubaustrecken, der bei vielen Prestigeprojekten oft sehr niedrig
liegt.

% spezifisches Problem
Die vorliegenden Ausarbeitung untersucht die Kosten zum Bau von Infrastruktur im
Schienenpersonennahverkehr (SPNV). Die gewonnen Ergebnisse können genutzt
werden, um die Kosten von Stadt- und Straßenbahnprojekten genauer einzuschätzen
und zu bewerten.

% herangehens weise, Hauptaussagen
Zur Berechnung eines geeigneten Kostenmaßes werden unterschiedliche
Neubaustrecken vergliche und ausgewertet. Es wird versucht geeignete Kennwerte
zu suchen mittels linearer Regression die gewonnen Daten für relevante
Streckenmerkenmale zur generalsieren. Die Auswertung erfolgt in
unterschiedlichen Streckenklassen. Hierzu werden die Strecken in Straßenbahnen
und U-Bahnen unterteilt und auf ihre Parameter untersucht. Zum historischen
Vergleich werden zudem Neubauten aus der Zeit 1900 in Darmstadt hinzugezogen.


% Struktur der restlichen Arbeit
Im Rest der Arbeit erläutern wir zunächst unseren Untersuchungsansatz und die
Datenerhebung genauer. Im nächsten Abschnitt beschreiben wir die Daten, die wir
für verschiedene Strecken gesammelt haben. Anschlißend versuchen wir diese Daten
in Relation zu setzen, sowie ein Preismodell für Nahverkehrsinfrastruktur zu
finden. Im letzten Abschnitt fassen wir unsere Ergebnisse zusammen.
