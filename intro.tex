\chapter{Einleitung}

% schema: http://www.cs.columbia.edu/~hgs/etc/intro-style.html

% Motivation für das Problemfeld
Die mediale Berichterstattung über Infrastrukturprojekte ist geprägt, vom Widerstand der
Bürger gegen diese. Als bisheriger Höhepunkt dessen ist sicherlich die politische
Auseinadersetzung sowie die Medientrubel rund um die Proteste gegen \emph{Stuttgart 21} im
Jahr 2010 zu nennen. Im Mittelpunkt solcher Konflikte steh neben den raumgestallterischen
oft auch der finanziellen Aspekt. Wichtig für diesen ist vorallem das Verhältnis zwischen
Herrstellungskosten und entstehenden Nutzen von Neubaustrecken, der bei vielen
Prestigeprojekten oft sehr niedrig liegt. Oft auch wegen einer viel zu geringen
Einschätzung der Kosten in der Plannungsphase.

% spezifisches Problem
Die vorliegenden Ausarbeitung untersucht die Kosten zum Bau von Infrastruktur im
Schienenpersonennahverkehr (SPNV). Die gewonnen Ergebnisse können genutzt werden, um die
Kosten von Stadt- und Straßenbahnprojekten einzuschätzen und zu bewerten.

% herangehens weise, Hauptaussagen
Zur Berechnung eines geeigneten Kostenmaßes werden unterschiedliche
Neubaustrecken verglichen und ausgewertet. Es wird versucht geeignete
Kennwerte zu suchen und mittels linearer Regression die gewonnen Daten
für relevante Streckenmerkenmale zu generalsieren. Die Auswertung erfolgt in
Neubaustrecken verglichen und ausgewertet. Es wird versucht geeignete Kennwerte
zu ermitteln und  mittels linearer Regression die gewonnen Daten für einige
Streckenmerkenmale zur generalsieren. Die Auswertung erfolgt in
unterschiedlichen Streckenklassen. Hierzu werden die Strecken in Straßenbahnen
und U-Bahnen unterteilt und auf ihre Parameter untersucht. Zum historischen
Vergleich werden zudem Neubauten in Damrstadt aus der Zeit um 1900 hinzugezogen.

% Struktur der restlichen Arbeit
Im Rest der Arbeit erläutern wir zunächst unseren Untersuchungsansatz und die
Datenerhebung genauer. Im nächsten Abschnitt beschreiben wir unsere Rechercheergebnisse
für die verschiedene Strecken. Anschließend versuchen wir diese Daten in Relation zu
setzen, sowie ein Preismodell für Nahverkehrsinfrastruktur zu finden. Im letzten Abschnitt
fassen wir unsere Ergebnisse zusammen und zeigen Schwächen unserer Methode auf.
