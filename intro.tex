\section{Einführung}

In der vorliegenden Ausarbeitung werden die Kosten zum Bau von Infrastruktur im Schienenpersonennahverkehr (SPNV) untersucht. Dazu werden unterschiedliche Streckenneubauten in verschiedenen Staaten verglichen, ausgewertet und der Durchschnitt gebildet.

Die Auswertung erfolgt in unterschiedlichen Streckenklassen. Hierzu werden die Strecken in Straßenbahnen und U-Bahnen unterteilt und auf ihre Parameter untersucht.

Zum historischen Vergleich werden zudem Streckenneubauten aus Darmstadt von 1897 bis 1903 ausgewertet. Diese gehen nicht in die Durchschnittswerte ein. 

Einige Strecken sind bisher nur projektiert, weshalb es hierzu nur Kostenschätzungen gibt.

\subsection{Defintion}

Untersucht werden Strecken, die wahlweise nach BOStrab oder vergleichbaren Verordnungen gebaut und betrieben werden.

Die Strecken werden in die Kategorien Straßenbahn und U-Bahn unterteilt. Als U-Bahn zählen diejenigen Strecken, welche kreuzungsfrei angelegt wurden und über Tunnel- oder Hochbahnanteil verfügen. Als Straßenbahn werden alle übrigen Strecken erfasst, sie können jedoch auch mit Tunnels trassiert sein.

Lorem ipsum \cite{amtsblattFRzaehringen}
