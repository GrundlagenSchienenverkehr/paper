\chapter{Grundlagen}

Ziel dieser Arbeit ist es, die durchschnittlichen Kosten von Stadt- und
Straßenbahnen zu ermitteln. Die Aufgabenstellung war damit sehr weit gefasst. In
diesem Abschnit beschreiben wir, mit welchem Schwerpunkt und welcher Mehtode wir
das Problem untersucht haben.

Der Begriff der Stadt- und Straßenbahen kann verschieden intterpretiert
werden. Als Straßenbahnen oder Tram versteht man üblicherweise Bahnen die nach
BOStrab \cite{bostrab} und nicht kreuzungsfrei mit dem Individualverkehr gebaut
und betrieben werden. Hierzu abzugrenzen sind unterirdische Bahnen oder
Hochbahnen die in der Regel auch nach BoStrab betrieben werden, aber
kreuzungsfrei verkehren. Da da Ingenieursbauten bei letzteren meist im gesamten
Streckenverlauf errichtet werden müssen, betrachten wir sie gertrennt von den
Straßenbahnen. Der Einfachheit halber werden wir sowohl die unterirdischen, als
auch die Hochbahnen im weiteren Verlauf der Arbeit als \emph{U-Bahnen}
zusammenfassen. Auch bei den Straßenbahnen gibt es starke Abweichungen in Bezug
auf Brücken und Tunnelanteile, auf diese werden wir in der Auswertung eingehen.

Bahnen die in der Stadt nach EBO \cite{ebo} betreiben werden, wie zum Beispiel
die Berliner S-Bahn, bleiben in dieser Arbeit aussenvor. Denn sowohl ein
Verlgeich mit U-Bahnen als auch mit Straßenbahnen wäre nicht gerechtfertigt. Wir
empfehlen diese eher mit regulären Eisenbahnstrecken im stdätischen Gebiet zu
vergleichen.

Neben der Betrachtung von Bahnen in Deutschland, haben wir auch einige Bahnen in
Europa und dem Rest der Welt untersucht, bei der Berechnung des Mittelwertes
wollen wir dieses allerdings nicht berücksichtigen. Während in Europa zumindest noch
ansatzweise gleiche gesetzliche Regelungen herrsche, so ist die
Vergleichbarkeit allein durch Lohnunterschiede nicht mehr gegeben. Weitere,
ähnlich abweichende Faktoren sind ebenfalls zu erwarten.

Es gibt unterschiedliche Methoden die durchschnittlichen Kosten für den Bau von
Infrastruktur zu ermitteln. Zum einen könnte man einen konstruktiven,
laborähnlichen Ansatz wählen und zusammenstellen, welche Komponeten für die
durchschnittliche Stadt- und Straßebahnstrecke nötig sind. Andschließend können
die Arbeitzeiten der einzelnen Gewerke sowie die nötige Materialmengen
abgeschätzt werden. Zusammen mit den üblichen Tariflöhnen und Einkaufspreisen
lässt sich dann ein durchschnittlicher Preis berechnen. Nachteile dieser Methode
sind der sehr hohe Aufwand, mögliche saisonale und regionale Schwankungen in den
Preisen sowie die Frage, was genau eine durchschnittliche Strecke ist. Darüber
hinaus geben sich die Hersteller erfahrungsgemäß eher bedeckt, was Preise von
Fertigkomponten wie Signalen oder Ähnlichem betrifft. Nicht zuletzt werden in
der Realität die Preise für viele Komponten und Dienstleistungen nicht fix sein,
sondern für jeden Auftrag neu ausgehandelt.

Realistischer ist eine beobachtende Perspektive einzunehmen und vorhandene
Strecken bezüglich ihrer Kosten zu analysieren. Bei diesem Ansatz müssen vor
allem Daten über die Strecken erhoben werden. Im Idealfall würde man dazu
Einsicht in tatsächlich erfolgte Zahlungen und Detailpläne nehmen und dann
kompontenweise die Preise verschiedener Strecken vergleichen. Eine solche
Detailtiefe scheint aber zumindest bei Recherche über Fachmagazine und das
Internet nicht möglich. Sie ist auch für eine repräsentativ großem Menge an
Strecken sehr aufwendig. Da es sich im Rahmen unserer Möglichkeiten schon als
schwierig erwies, die Kosten für einen einzelnen Haltestelle zu ermitteln, haben
wir uns stattdessen darauf konzentriert, einge wichtige Daten für die Strecken
sowie den Gesamtpreis zu sammeln. Damit haben wir anschließend Kennzahlen wie
zum Beispiel Gesamtkosten pro Halt berechnet und diese dann in Relation
gesetzt. Diese Verfahren ist zwar praktikabel, jedoch stärker fehlerbehaftet. Eine
genaue Analyse der Fehlerquellen findet sich im vierten Abschnitt der Arbeit.

Um die verschiedenen Strecken innerhalb der Untersuchung vergleichen zu können,
müssen die zu ermittelnen Merkmale festgelegt werden. Dabei haben wir uns auf
folgende geeinigt, da sie sich meist in Fachzeitschriften,
Planfeststellungsverfahren oder auf den Websiten der Verkehrsbetriebe finden
lassen:

\newcounter{fnnumber}
\begin{itemize}
\item Streckenlänge
\item Tunnelanteile
\item Brückenanteil
\item Gleislänge \footnote{soweit vorhanden}\setcounter{fnnumber}{\thefootnote}
\item Weichenanzahl \footnotemark[\thefnnumber]
\item Anzahl der Haltepunkte und Bahnhöfe
\item tatsächliche bzw. geschätzte Gesamtkosten
\item Jahr der Fertigstellung
\end{itemize}

Wir vermuten, dass der wichtigste Faktor die Streckenlänge ist, wobei Brücken
bzw. Tunnelabschnitte teurer sind. Da Bahnhöfe und Haltepunkte zusätzliche
Ausstattungen und Gebäude erfordern, erscheinen uns diese auch als relevant für
die Gesamtkosten. Das Jahr der Fertigstellung wird benötigt, um die Inflation
beim Vergleich der Preise herauszurechnen.

Um die gennanten Daten zuerheben, haben wir vorallem Fachzeitschriften und
Artikel in Tageszeitungen benutzt. Bei einigen Verkehrsbetrieben haben wir auch
direkt Anfragen gestellt, leider wurden nicht alle beantwortet. Die Ergebnisse
unserer Recherche finden sich im nächsten Abschnitt.
