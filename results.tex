\section{Ergebnisse}

\begin{comment}

Gliederung meiner Streckenbeschreibung (Yves)

* Übersicht
  * Umfeld des Bauprojekts
  * von wo nach wo
  * historische Einordnung

* konkrter Streckenverlauf
  * detalierte Daten (weichen etc ..)


* Probleme
  * bautechnisch
  * gesellschaftlich

* Zeitplan

* Kosten
\end{comment}


\subsection{U-Bahnprojekte}

\subsubsection*{Berlin - U5 Lückenschluß}

Die U-Bahnlinie 5 fährt auf dem Abschnitt zwischen Alexanderplatz und
Friedrichsfelde schon seit 1930 \cite{baukammerHeftU5}. Schon im Vorfeld der
Eröffnung dieses Abschnittes, in den 20er Jahren, gab es Plannungen die Linie
Richtung Westen weiter zu verlängern \cite{bvgWebsiteU5}. Ein Vorgriff darauf
war die Eröffnung des Teilstücks U55 zwischen Brandenburger Tor (ehemals Unter
den Linden) und Hauptbahnhof. Die Schaffung dieser Insellösung begründet wurde
nötig durch Verträgen zwischen dem Land Berlin und dem Bund anläßlich der
Verlegung des Regierungssitzes \cite{hauptstadtvertrag}. Um die entstande Lücke
zu schließen plant und baut die BVG eine Neubaustrecke zwischen Alexanderplatz
und Brandenburger Tor. Die ersten Bauarbeiten wurden 2010 begonnen,mit dem
Abschluss wird bisher 2019 gerechnet \cite{bvgWebsiteU5plan}. In unserer Arbeit
widmen wir uns auch der Untersuchung dieser Strecke.

Die neue Strecke beginnt am Alexanderplatz und verläuft vor dem Berlinerrathaus,
um dort auch schon die erste Station Berliner Rathaus zuerreichen
\cite{baukammerHeftU5} \cite{bvgWebsiteU5} Unmittelbar im Anschluß an den
Alexanderplatz wird ebenfalls eine Gleiswechselanlage erreichtet. Im weiteren
ebenfalls unterirdischen Streckenverlauf wird die Spree von der Station
Museumsinsel unterquert. Ab dort verläuft die Strecke entlang der Straße Unter
den Linden, wo sie im gleichnamigen Kreuzungsbahnhof mit der Linie U6 hält und
anschlißend an der Station Brandeburger Tor in die U55 endet. Die Gesamtläng der
Strecke beträgt damit etwa 2,2 km.

Der Bau wird komplett im Schildvortrieb ausgeführt, was durch den Sandboden vor
Ort erschwert wird \cite{baukammerHeftU5}. Des weiteren ist liegt die Strecke
mit bis zu 25 m vergleichsweise tief. Dies wird nötig, da sowohl die Spree als
auch die U6 unterquert werden müssen. Ersteres erfordert auch die Konstruktion
des Bahnhofs Museumsinsel im bergmännischen Verfahren, während die anderen
Bahnhöfe im Schachtverfahren gebaut werden können. Neben den baulichen
Schwierigkeiten ist auch der Nutzen des Projektes umstritten \cite{ftdU5}. Da
die Strecke weitgehend parralell zu Stadtbahn verläuft und auch der
Kosten-Nutzenfaktor bis Turmstraße nur kanpp über 1 liegt. Insgesamt werden die
Kosten für die Strecke auf 435 Mio. Euro geschätzt \cite{berlinWebsiteU5}. Wobei
sich zu dieser offiziellen Schätzung auch kritische Kommentar finden, die von
noch deutlich höhren Kosten ausgehen \cite{ftdU5}.

\subsubsection*{Hamburg - U4}

Die U4 in Hamburg ist eine 2012 neugeschaffene Linie, die sich allerdings die
meisten ihrere Haltstelle mit der U2 teilt. Ihre Aufgabe ist der Anchluss der
Hafencity and das Hamburger Hochbahnnetz \cite{keuchelHamburg}. Um dies
umzusetzen, wurde abzweigend von der vorhanden U2 Station Jungfernstieg, eine neue
Strecke gebaut, die im Rahmen dieser Arbeit ebenfalls untersucht wird. Dabei
werden wir uns vorallem auf den schon fertigen Abschnitt konzentieren. Die
im baubefindliche Verlängerung bleibt zu nächst außen vor.

Die Neubaustrecke hat eine Gesamtlänge von ca. 4 km \cite{keuchelHamburg}. Sie
verläuft von Jungfernstieg ausgehend unter der U3 hindurch, um dann anschließend
das Hafenbecken zu unterqueren. Ihre beiden Haltestellen Überseequatier und
HafenCity Universität liegen dann beide auf dem Gebiet der Hafencity. Beide
Tunnelstationen konnten in offener Bausweise errichtet werden. Alle anderen
Streckenteile verlaufen ebenfalls unterirdisch und wurden im
Schildvortriebverfahren gebaut.

Neben den bautechnischen Schwierigkeiten der Hafenunterquerung hat auch dieses
Projekt mit Kritik auf verkehrspolitscher Ebene zu kämpfen. So wird die
unteridische Streckenführung kritisiert, da sie teurer sei, als eine
Hochbahnlösung. \cite{hamburgerAbendblattu4}.

\subsection{Straßenbahnprojekte}

\begin{landscape}
\include{results_table}
\end{landscape}

%%% Local Variables:
%%% mode: latex
%%% TeX-master: "main"
%%% End:
