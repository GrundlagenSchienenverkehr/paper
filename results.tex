\section{Ergebnisse}

\begin{comment}

Gliederung meiner Streckenbeschreibung (Yves)

* Übersicht
  * Umfeld des Bauprojekts
  * von wo nach wo
  * historische Einordnung

* konkrter Streckenverlauf
  * detalierte Daten (weichen etc ..)


* Probleme
  * bautechnisch
  * gesellschaftlich

* Zeitplan

* Kosten
\end{comment}



\subsection{Berlin}
\subsubsection*{Lückenschluss der U5}

Die U-Bahnlinie 5 fährt auf dem Abschnitt zwischen Alexanderplatz und
Friedrichsfelde schon seit 1930 \cite{bkhU5}. Schon im Vorfeld der
Eröffnung dieses Abschnittes, in den 20er Jahren, gab es Planungen, die Linie
Richtung Westen weiter zu verlängern \cite{bvgWebsiteU5}. Ein Vorgriff darauf
war die Eröffnung des Teilstücks U55 zwischen Brandenburger Tor (ehemals Unter
den Linden) und Hauptbahnhof. Die Schaffung dieser Insellösung wurde
nötig durch Verträge zwischen dem Land Berlin und dem Bund anlässlich der
Verlegung des Regierungssitzes \cite{hauptstadtvertrag}. Um die entstande Lücke
zu schließen, plant und baut die BVG eine Neubaustrecke zwischen Alexanderplatz
und Brandenburger Tor. Die ersten Bauarbeiten wurden 2010 begonnen, mit dem
Abschluss wird bisher 2019 gerechnet \cite{bvgWebsiteU5plan}. In unserer Arbeit
widmen wir uns auch der Untersuchung dieser Strecke.

Die neue Strecke beginnt am Alexanderplatz und verläuft vor dem Berliner Rathaus,
um dort auch schon die erste Station Berliner Rathaus zuerreichen
\cite{bkhU5} \cite{bvgWebsiteU5} Unmittelbar im Anschluß an den
Alexanderplatz wird ebenfalls eine Gleiswechselanlage errichtet. Im weiteren
(ebenfalls unterirdischen) Streckenverlauf wird die Spree von der Station
Museumsinsel unterquert. Ab dort verläuft die Strecke entlang der Straße Unter
den Linden, wo sie im gleichnamigen Kreuzungsbahnhof mit der Linie U6 hält und
anschließend an der Station Brandeburger Tor in die U55 endet. Die Gesamtläng der
Strecke beträgt damit etwa 2,2 km.

Der Bau wird komplett im Schildvortrieb ausgeführt, was durch den Sandboden vor
Ort erschwert wird \cite{bkhU5}. Desweiteren liegt die Strecke
mit bis zu 25 m vergleichsweise tief. Dies wird nötig, da sowohl die Spree als
auch die U6 unterquert werden müssen. Ersteres erfordert auch die Konstruktion
des Bahnhofs Museumsinsel im bergmännischen Verfahren, während die anderen
Bahnhöfe im Schachtverfahren gebaut werden können. Neben den baulichen
Schwierigkeiten ist auch der Nutzen des Projektes umstritten \cite{ftdU5}, da
die Strecke weitgehend parralell zu Stadtbahn verläuft und auch der
Kosten-Nutzenfaktor bis Turmstraße nur knapp über 1 liegt. Insgesamt werden die
Kosten für die Strecke auf 435 Mio. Euro geschätzt \cite{bwwwU5}. Wobei
sich zu dieser offiziellen Schätzung auch kritische Kommentare finden, die von
noch deutlich höhren Kosten ausgehen \cite{ftdU5}.

\subsubsection*{Straßenbahn zum Hauptbahnhof}

bla

\subsection{Hamburg - U4}

Die U4 in Hamburg ist eine 2012 neugeschaffene Linie, die sich allerdings die
meisten ihrer Haltstellen mit der U2 teilt. Ihre Aufgabe ist der Anchluss der
Hafencity an das Hamburger Hochbahnnetz \cite{keuHH}. Um dies
umzusetzen, wurde abzweigend von der vorhanden U2-Station Jungfernstieg eine neue
Strecke gebaut, die im Rahmen dieser Arbeit ebenfalls untersucht wird. Dabei
werden wir uns vor allem auf den schon fertigen Abschnitt konzentieren. Die
im Bau befindliche Verlängerung bleibt zunächst außen vor.

Die Neubaustrecke hat eine Gesamtlänge von ca. 4 km \cite{keuHH}. Sie
verläuft von Jungfernstieg ausgehend unter der U3 hindurch, um dann anschließend
das Hafenbecken zu unterqueren. Ihre beiden Haltestellen Überseequartier und
HafenCity Universität liegen dann beide auf dem Gebiet der Hafencity. Beide
Tunnelstationen konnten in offener Bausweise errichtet werden. Alle anderen
Streckenteile verlaufen ebenfalls unterirdisch und wurden im
Schildvortriebverfahren gebaut.

Neben den bautechnischen Schwierigkeiten der Hafenunterquerung hat auch dieses
Projekt mit Kritik auf verkehrspolitscher Ebene zu kämpfen. So wird die
unterirdische Streckenführung kritisiert, da sie teurer sei, als eine
Hochbahnlösung. \cite{hamburgerAbendblattu4}.

\subsection{Stuttgart}

Die Geschichte der Stuttgarter Straßenbahn geht zurück bis ins Jahr 1868. In den 1960er- und 1970er-Jahren wurden die damals oberirdisch liegenden Streckenabschnitte im Innenstadtbereich unter die Erde verlegt.
In den 1970er-Jahren gab es Planungen zum Aufbau eines U-Bahn-Netzes, es wurde sich jedoch dafür entschieden, das (meterspurige) Straßenbahnnetz sukzessive zur (normalspurigen) Stadtbahn umzubauen. Im Jahr 1985 wurde die erste Stadtbahnlinie am südlichen Stadtrand eröffnet, Ende 2007 wurde der Betrieb auf der Meterspur eingestellt \cite{SSBgeschichte}. Die im Folgenden vorgestellten Strecken der Linie U15 sind Teil dieses Umbaus.

\subsubsection*{U15 nach Ruhbank}



\subsubsection*{U15 nach Stammheim}

\subsubsection*{U12 nach Fasanenhof}


\subsection{Magdeburg}


\subsection{Nürnberg}


\subsection{Bremen}


\subsection{Freiburg}


\subsection{Darmstadt}

\subsubsection*{Darmstadt (historisch)}

Darmstadt ist eine mittelgroße Stadt in Hessen. Um 1900 lebten rund 70.000 Menschen in
Darmstadt. Daher war die Stadt, die zugleich Sitz des Großherzogtums Hessen-Darmstadt
war, bestrebt, eine Straßenbahn einzurichten. Das erste Netz entstand somit in den
Jahren 1897 bis 1903 und verband die Viertel der wohlhabenden Bevölkerung, die
Innenstadt und die Bahnhöfe miteinander.

Zur Inbetriebnahme der Darmstädter Straßenbahn wurden 1897 zwei Strecken errichtet,
welche sich zwischen dem Schloß und dem Pädagog die Strecke teilten. Die erste Linie
verlief von der Station "Böllenfalltor", an welcher sich der Betriebshof befand, entlang
einer damals nur schwach besiedelten Villenkollonie über den Friedhof, das Pädagog und
das Schloß (wo Anschluss an drei Vorort-Dampfstraßenbahnlinien bestand) zu den
(Haupt)Bahnhöfen. Die zweite Linie wurde von der Taunusstraße (Brauereiviertel) über das
Schloß und das Pädagog bis zur Hermannstraße. Letztere Station endete somit an der
Stadtteilgrenze zu einem der wohlhabenderen Darmstädter Vierteln.

In den folgenden Jahren wurden verschiedene Streckenergänzungen vorgenommen. So wurde
1900 die Strecke zur Hermannstraße bis zur Ludwigshöhstraße verlängert und ershcloss
somit das Viertel Bessungen. 1902 wurde die Strecke zur Taunusstraße bis zur Fasanerie
verlängert und erschloss somit ein Villenviertel und ein Ausflugsziel. 1903 folgten die
Errichtung einer neuen Strecke von den Bahnhöfen zum Schloßgartenplatz, wodurch das
Paulusviertel erschlossen wurde, in welchem viele Beamte wohnhaft waren. Ebenfalls 1903
wurde eine Strecke vom Schloß in die südliche Innenstadt errichtet, um eine Konkurrenz zu
einer der Dampfstraßenbahnstrecken aufzubauen.

Die Streckenbauten von 1897 umfassten 6,4 km Strecke, von denen 0,9 km zweigleisig
ausgeführt wurden. Die eingleisigen Streckenabschnitte hatten insgesamt 8 Ausweichstellen.
Die Baukosten betrugen 287.000 Mark.

Die Streckenaus- und -neubauten von 1899 bis 1903 umfassten insgesamt 5,98 km, von denen
0,64 km zweigleisig ausgeführt wurden. Es wurden 4 Ausweichstellen vorgesehen. Die Kosten
für die Erweiterung beliefen sich auf 1,2 Mio. Mark, wobei hierin auch 16 neue Fahrzeugen
enthalten sind. Da die 1897 beschafften 18 Fahrzeuge geringerer Länge und einfacherer
Technik 262.000 Mark kosteten, was etwa der Hälfte der Gesamtkosten entspricht, ist davon
auszugehen, dass etwa die Hälfte der Investitionen in den Ausbau auf die
Fahrzeugbeschaffung entfiel. Somit kostete der Ausbau ca. 600.000 Mark.

Die Strecken wurden weitestgehend in Straßenlage in einfacher Bauform ausgeführt. Dabei
wurde der Untergrund verdichtet, die Schienen darauf gelegt (und mittels Querstreben in
ihrer Spurweite gefestigt) und wahlweise mit Sand aufgefüllt oder gepflastert. Die
Fahrleitung wurde durch einfache Wandrosetten an den umliegenden Häusern befestigt,
sofern genügend hiervon vorhanden waren.

Bei der Erstausstattung 1897 wurden, abweichend von der vogehend beschriebenen
Standardbauform, 1,64 km als besonderer Bahnkörper ausgeführt und mit Oberleitungsmasten
versehen.
Bei den Ausbauten von 1902 wurden abweichend etwa 1,29 km mit Oberleitungsmasten und
Querverspannung ausgeführt.

Aus der Bauphase sind keine Probleme bekannt geworden. Lediglich bei der Trassierung von
1897 war ein Häuserblock im Weg, welcher abgetragen werden musste. Über
Entschädigungszahlungen ist nichts bekannt, da sich die fragliche Stelle jedoch in der
armen Innenstadt befand ist davon auszugehen, dass keine Entschädigungszahlungen in
nennenswertem Umfang gezahlt wurden.

Der Bau der ersten Strecken erfolgt zwischen April und Novemver 1897.

\subsubsection*{Darmstadt (aktuell)}

Darmstadt hat derzeit etwa 140.000 Einwohner. Die Dampfbahnstrecken wurden bis 1926
elektrifiziert. Zudem gab es diverse Streckenneubauten und -einstellungen. Das Netz
umfasste 2009 somit etwa 40 km Länge.

Um die Verkehrssituation im Norden Darmstadts zu verbessern, wurde von Dezember 2006
bis Juni 2009 die Strecke zwischen Merck und Arheilgen gesperrt und grundlegend saniert.
Bis auf die Trassenführung und die Haltestellenlage bleib jedoch nichts erhalten, weshalb
die Strecke als Neubau gewertet werden kann. Ab 2009 bis August 2011 erfolgte die
Streckenverlängerung bis zum nördlichen Bebauungsrand Arheilgens.

Der erste Bauabschnitt umfasst 1,3 km Länge und kostete 23,3 Mio. Euro. Der zweite
Bauabschnitt ist 1,2 km lang und kostete 15,2 Mio. Euro. Beide Abschnitte wurden
zweigleisig und in Straßenlage ausgeführt. Es gibt keine nennenswerten Ingenieursbauwerke.

Im ersten Bauabschnitt gab es Probleme auf Grund von alten, unbekannten
Versorgungsleitungen im Erdreich, weshalb sich die Gesamtkosten für beide Bauprojekte von
ursprünglich 26,9 Mio. Euro auf 38,5 Mio. Euro erhöht haben.

Links für Quellen:
http://www.echo-online.de/region/darmstadt/Noch-ein-Stueck-Neue-Wege;art1231,944296
http://www.echo-online.de/region/darmstadt/Strassenbahn-Projekt-Neue-Wege-fuer-Arheilgen-abgeschlossen;art1231,2051268
http://www.echo-online.de/region/darmstadt/Festtag-fuer-Arheilgen-Die-Strassenbahn-rollt-wieder;art1231,2062370

\subsubsection*{Darmstadt (geplant)}

Zur Erschließung des Campus Lichtwiese der TU-Darmstadt ist geplant, bis etwa 2017 eine neue
Straßenbahnstrecke zu errichten. Dazu wurden 2013 zwei Trassenvarianten untersucht.

Variante I umfasst eine Länge von 2,05 km, wovon 1,2 km in Straßenlage und 0,85 km als
eigener Bahnkörper zu errichten wären. Die Kosten für diese Variante werden auf 23,58 Mio.
Euro geschätzt.

Variante II umfasst eine Länge von 1,33 km, die vollständig auf eigenem Bahnkörper verlaufen.
Die Baukosten werden bei dieser Variante auf 8,32 Mio. Euro geschätzt.

Der sich aus Variante II ergebenden Durchnittswert von 6,3 Mio. Euro/km liegt deutlich unter
dem Durchnittswert, welcher sich aus dieser Arbeit ergibt. Es liegt also nahe, dass dieser
Wert nicht realitätsnah ist. Daher wird er in der weiteren Ausarbeitung nicht mehr
berücksichtigt!

Links für Quellen:
https://darmstadt.more-rubin1.de/beschluesse_details.php?vid=221405100167&nid=ni_2013-Mag-324&suchbegriffe=&select_gremium=select_gremium&datum_von=&datum_bis=&entry=&status=1
http://www.echo-online.de/region/darmstadt/Entscheidung-fuer-Strassenbahn-Abzweig-zur-Lichtwiese;art1231,4069073

\subsection{Bergen}

Bergen ist eine etwa 270.000 Einwohner zählende Stadt in der Norwegischen Provinz Hordaland.
und um Bergen befindet sich hügeliges, wetestgehend unbewohntes Gelände. Bereits von 1897 bis
1966 verfügte Bergen über eine elektrische Straßenbahn.

Bergen hat 2010 eine Stadtbahn von Byparken (Innenstadt) nach Nestun eingerichtet. Die Strecke
hat eine Länge von 9,8 km, von denen 2,63 km in 4 Tunneln liegen (Fageråstunnel (663 m),
Slettebakkstunnel (412 m), Fantofttunnel (1107 m), Tveiteråstunnel (443 m)). Die gesamte
Strecke ist 2-gleisig ausgebaut und verfügt über vereinfachte Zugsicherungs- und Signaltechnik
mit Vorrangschaltung gegenüber dem Straßenverkehr.

Die Strecke wurde von Januar 2008 bis Sommer 2010 errichtet. Etwa 5,9 km der Strecke liegen
gebündelt mit Straßen, jedoch auf eigenem Bahnkörper als Rasengleis bzw. befahrbar für
Rettungsfahrzeuge. Die restlichen 3,9 km sind S-Bahn-ähnlich angelegt und beinhalten auch die
vier Tunnel. Der minimale Radius der Strecken beträgt 25 m, der maximale Neigungskoeffizient
liegt bei 6%.

Die Stadtbahn ist Teil des 500 Mio. Euro umfassenden "Bergen-Programm" für Umwelt, Entwicklung
und Verkehr, welches von 2002 bis 2012 lief. Bis 2015 sind hier weitere 143 Mio. Euro
eingeplant, unter anderem für eine Erweiterung des neuen Stadtbahnnetzes.

Die Gesamtkosten für das Projekt wurden auf 140 Mio. Euro projektiert, von denen etwa
12 x 2,5 Mio. Euro für Fahrzeuge abzuziehen sind. Damit ergeben sich Baukosten von ca. 110
Mio. Euro.

Seit Januar 2011 wird an der ersten Verlängerung nach Lagunen (3,6 km) gebaut. Sie wurde am
21. Juni 2013 eröffnet.

Der dritte Bauabschnitt bis zum Flughafen soll im August 2013 begonnen und bis 2015 fertiggestellt
werden. Er weist eine Länge von 6,5 km auf.

\subsection{Århus}

Århus ist die mit etwa 320.000 Einwohnern zweitgrößte Stadt in Dänemark.Für 2016 plant man,
in Århus ein Stadtbahnnetz einzurichten, welches die Stadt und das Umland verbinden soll.

Die geplante Neubaustreckenlänge umfasst 12,0 km. Diese werden auf eigenem Bahnkörper, jedoch
mit der Straße gebündelt ausgeführt. Die gesamte Strecke wird zweigleisig ausgeführt.

Zur Erschließung des Umlandes wird eine etwa 95 km lange Eisenbahnstrecke für den Betrieb mit
Tram-Train-Fahrzeugen (Stadtbahnwagen) angepasst. Tunnelneu- oder -umbauten sind nicht
notwendig.

Die Kosten für den Straßenbahnabschnitt werden mit 500 Mio. Kronen angegeben, was etwa 65 Mio.
Euro entspricht.

\subsection{Toulouse}


\subsection{Los Angeles}


\subsection{Dubai}


\subsection{Brescia}






\begin{landscape}
\include{results_table}
\end{landscape}

%%% Local Variables:
%%% mode: latex
%%% TeX-master: "main"
%%% End:
