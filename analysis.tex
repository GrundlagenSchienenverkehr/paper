\chapter{Auswertung}


% Tabelle mit durchschnitts Kosten und abweichung der bahnen davon
% Diagramm (Balken) mit Kosten pro km/station

\section{Straßenbahnen}

Aus den gefundenen Daten haben wir die Kosten pro Kilometer sowie pro Halt berechnet:

\input{analysis_table_trams}

Es zeigen sich sehr große Abweichungen zwischen den verschiedenen Projekten. Anders als vermutet liegen jedoch die Kosten der ausländischen Projekte in einem ähnlichen Bereich wie die der deutschen.

Trotz der breiten Streuung lässt sich feststellen, dass bei gut der Hälfte der Projekte die relativen Kosten zwischen sechs und 13 Mio. Euro pro Kilometer liegen.

Im folgenden versuchen wir Besonderheiten der aus diesem Rahmen fallenden Projekten zu finden, die Gründe für die höheren Kosten sein könnten.

\begin{itemize}

    \item \textbf{Freiburg-Zähringen:} Hier war der Neubau einer Brücke über einen Bach sowie der Bau einer zweiten Brücke der Güterbahn, die die Strecke kreuzt, nötig. Um für die restliche Strecke einen Kilometerpreis von 13{\ }Mio.{\ }Euro zu erhalten, muss man für die beiden Bauwerke einen Preis von etwa drei Millionen Euro ansetzen. Das erscheint uns denkbar.
    \item \textbf{Darmstadt-Arheilgen (BA 1):} Hier gab es Probleme mit alten, unbekannten Versorgungsleitungen, wodurch sich das Gesamtprojekt von 26,9{\ }Mio.{\ }Euro auf 38,5{\ }Mio.{\ }Euro erhöhte. Die ursprünglich angenommenen Kosten ergeben einen Kilometerpreis von 10,8{\ }Mio.{\ }Euro, was wieder sehr gut im Rahmen liegt.
    \item \textbf{Stuttgart-Stammheim:} Wie erwartet, hat der Tunnel die Kosten des Projekts sehr weit hochgetrieben. Nimmt man jedoch für das einen Kilometer lange Tunnelstück Kosten von etwa 80{\ }Mio.{\ }Euro und für den knapp zwei Kilometer langen oberirdischen Teil von knapp 13{\ }Mio.{\ }Euro an, so ergeben sich als Summe die 105{\ }Mio{\ }Euro, die das Projekt tatsächlich gekostet hat. Die beiden angenommenen Werte liegen jeweils am oberen Ende des von uns für oberirdische bzw. unterirdische (siehe unten) Strecken angenommenen Rahmens.
    \item \textbf{T1 in Toulouse:} Hier musste eine Brücke über eine Autobahn neu gebaut werden; diese alleine kann jedoch nicht den großen Kostenunterschied erklären. Da dies die erste Tramlinie in Toulouse ist, vermuten wir, dass Kosten für allgemeine Infrastruktur wie Betriebshöfe in dem genannten Preis enthalten sind. Da schon zwei Streckenerweiterungen in Bau sind, ist auch zu vermuten, dass diese Infrastruktur gleich größer ausgelegt wurde. Da die Toulouser U-Bahn (trotz fahrerlosem Betrieb) eher günstiger war als die deutschen U-Bahnen, schließen wir landestypische Rahmenbedingungen als Ursache aus.
    %TODO darmstadt billiger weil alt bla

\end{itemize}

\section{U-Bahnen}

\input{analysis_table_subway}


\input{analysis_regression_tram}

\input{analysis_regression_subway}

\includegraphics[width=\textwidth]{images/costsPerKmSubway}
\includegraphics[width=\textwidth]{images/costsPerKmTram}

Ergebnis der Regression bei Trams: Sreckenkilometer für \tramnormal und Tunnelkilometer für \tramtunnel.
Ergebnis der Regression bei UBahn: Sreckenkilometer für \subwaynormal und Tunnelkilometer für \subwaytunnel.

