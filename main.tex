\documentclass[oneside,a4paper,11pt,german]{report}

% vim: tw=80
% -*- mode: Latex; fill-column: 80; -*-

\usepackage[utf8]{inputenc}
\usepackage{graphicx}
\usepackage{verbatim}
\usepackage{lscape}
\usepackage{pdflscape}
\usepackage{babel}
\usepackage{url}
\usepackage{breakurl}
\usepackage[breaklinks]{hyperref}
\usepackage{footmisc}
\usepackage{bibgerm}
\usepackage{eurosym}
\usepackage{textcomp}
\usepackage{titlesec}

\title{Kosten von Infrastruktur \\ für Stadt- und Straßenbahnen}
\author{
  Brandes, Dario \and
  Müller, Yves \and
  Riecken, Robert \and
  Sigler, Marian \and
  Sulfrian, Alexander
}

\setlength{\parindent}{0em}
\setlength{\parskip}{.5em}

% break url at / and -
\def\UrlBreaks{\do\/\do-}

\begin{document}

\maketitle

\begin{comment}
Gliederung:

* Einleitung


* Theorie / Grundlagen
  * Definitionen: Kein EBO, Trennung U-Bahn/Straßenbahn (Tunnel vs. nein)
  * Methode
     * wie wurden die Zahlen versucht zu ermitteln
     * welche Kennwerte wurden wie berechnet (kosten/strecken km)
     * warum sind die Kennwerte sinvoll

* Ergebnisse
  * Vorstellung der Projekte
    * Beschreibung der Strecke
    * momentaner Baustand / Plannungsstand
  * Tabelle allen Ergbnissen

* Auswertung
  * Ermittlung von Höchsts- /  Durchschnittswerten Inland getrennt U-Bahnen/Straßenbahnen
    * tabellarisch
    * mit Hilfe von Diagrammen
  * Vergleich mit den Werten im Ausland
  * Fehleranalyse (welche Außreißer existieren warum)

* Zusammenfassung


TODO:

 * Karten für alle Strecken erstellen und einfügen.
 * Tabelle mit Quellenangaben generieren.

\end{comment}

\tableofcontents

% \titleformat{Überschriftenklasse}[Absatzformatierung]{Textformatierung}{Nummerierung}
%   {Abstand zwischen Nummerierung und Überschriftentext}
%   {Code vor der Überschrift}[Code nach der Überschrift]
\titleformat{\chapter}[hang]{\Huge\bfseries}{\thechapter\quad}{0pt}{}
\titleformat{\section}[hang]{\LARGE\bfseries}{\thesection\quad}{0pt}{}
\titleformat{\subsection}[hang]{\Large\bfseries}{\thesubsection\quad}{0pt}{}

% \titlespacing{Überschriftenklasse}{Linker Einzug}{Platz oberhalb}
%   {Platz unterhalb}[rechter Einzug]
\titlespacing{\chapter}{0pt}{-2em}{2em}

% sections as sperate files
\chapter{Einleitung}

% schema: http://www.cs.columbia.edu/~hgs/etc/intro-style.html

% Motivation für das Problemfeld
Die mediale Berichterstattung über Infrastrukturprojekte ist geprägt vom Widerstand der
Bürger gegen diese. Als bisheriger Höhepunkt dessen ist sicherlich die politische
Auseinadersetzung sowie die Medientrubel rund um die Proteste gegen \emph{Stuttgart 21} im
Jahr 2010 zu nennen. Im Mittelpunkt solcher Konflikte stehen neben den raumgestallterischen
oft auch die finanziellen Aspekte. Wichtig für diese ist vorallem das Verhältnis zwischen
Herrstellungskosten und entstehenden Nutzen von Neubaustrecken, der bei einigen
Prestigeprojekten sehr niedrig liegt, oft auch wegen einer viel zu geringen Einschätzung
der Kosten in der Plannungsphase.

% spezifisches Problem
Die vorliegenden Ausarbeitung untersucht die Kosten zum Bau von Infrastruktur im
Schienenpersonennahverkehr (SPNV). Die gewonnen Erkenntnisse können genutzt werden, um die
Kosten von Stadt- und Straßenbahnprojekten abzuschätzen und zu bewerten.

% herangehens weise, Hauptaussagen
Zur Berechnung eines geeigneten Kostenmaßes werden unterschiedliche
Neubaustrecken verglichen und ausgewertet. Es wird versucht, geeignete
Kennwerte zu ermitteln und mittels linearer Regression die gewonnenen Daten
für relevante Streckenmerkenmale zu generalsieren. Die Auswertung erfolgt in
Neubaustrecken verglichen und ausgewertet. Es wird versucht geeignete Kennwerte
unterschiedlichen Streckenklassen. Hierzu werden die Strecken in Straßenbahnen
und U-Bahnen unterteilt und auf ihre Parameter untersucht. Zum historischen
Vergleich werden zudem Neubauten in Damrstadt aus der Zeit um 1900 hinzugezogen.

% Struktur der restlichen Arbeit
Im weiteren Verlauf der Arbeit werden zunächst der Untersuchungsansatz und die
Datenerhebung genauer erläutert. Im nächsten Abschnitt werden die Rechercheergebnisse
für die verschiedene Strecken untersucht. Anschließend werden diese Daten in Relation
gesetzt sowie ein Preismodell zum Bau von Nahverkehrsinfrastruktur erarbeitet. Im letzten
Abschnitt werden die Ergebnisse zusammengefasst und mögliche Nachteile unserer Methode
aufgezeigt.


\chapter{Grundlagen}

Ziel dieser Arbeit ist es, die durchschnittlichen Kosten von Stadt- und
Straßenbahnen zu ermitteln. Die Aufgabenstellung war damit sehr weit gefasst. In
diesem Abschnit beschreiben wir, mit welchem Schwerpunkt und welcher Mehtode wir
das Problem untersucht haben.

Der Begriff der Stadt- und Straßenbahen kann verschieden intterpretiert
werden. Als Straßenbahnen oder Tram versteht man üblicherweise Bahnen die nach
BOStrab \cite{bostrab} und nicht kreuzungsfrei mit dem Individualverkehr gebaut
und betrieben werden. Hierzu abzugrenzen sind unterirdische Bahnen oder
Hochbahnen die in der Regel auch nach BoStrab betrieben werden, aber
kreuzungsfrei verkehren. Da da Ingenieursbauten bei letzteren meist im gesamten
Streckenverlauf errichtet werden müssen, betrachten wir sie gertrennt von den
Straßenbahnen. Der Einfachheit halber werden wir sowohl die unterirdischen, als
auch die Hochbahnen im weiteren Verlauf der Arbeit als \emph{U-Bahnen}
zusammenfassen. Auch bei den Straßenbahnen gibt es starke Abweichungen in Bezug
auf Brücken und Tunnelanteile, auf diese werden wir in der Auswertung eingehen.

Bahnen die in der Stadt nach EBO \cite{ebo} betreiben werden, wie zum Beispiel
die Berliner S-Bahn, bleiben in dieser Arbeit aussenvor. Denn sowohl ein
Verlgeich mit U-Bahnen als auch mit Straßenbahnen wäre nicht gerechtfertigt. Wir
empfehlen diese eher mit regulären Eisenbahnstrecken im stdätischen Gebiet zu
vergleichen.

Neben der Betrachtung von Bahnen in Deutschland, haben wir auch einige Bahnen in
Europa und dem Rest der Welt untersucht, bei der Berechnung des Mittelwertes
lassen wir dieses allerdings außen vor. Während in Europa zumindest noch
ansatzweise gleiche gesetzliche Regelungen herrschen, so ist die
Vergleichbarkeit allein durch Lohnunterschiede nicht mehr gegeben. Weitere,
ähnlich abweichende Faktoren sind ebenfalls zu erwarten.

Es gibt unterschiedliche Methoden die durchschnittlichen Kosten für den Bau von
Infrastruktur zu ermitteln. Zum einen könnte man einen konstruktiven,
laborähnlichen Ansatz wählen und zusammenstellen, welche Komponeten für die
durchschnittliche Stadt- und Straßebahnstrecke nötig sind. Andschließend können
die Arbeitzeiten der einzelnen Gewerke sowie die nötige Materialmengen
abgeschätzt werden. Zusammen mit den üblichen Tariflöhnen und Einkaufspreisen
lässt sich dann ein durchschnittlicher Preis berechnen. Nachteile dieser Methode
sind der sehr hohe Aufwand, mögliche saisonale und regionale Schwankungen in den
Preisen sowie die Frage, was genau eine durchschnittliche Strecke ist. Darüber
hinaus geben sich die Hersteller erfahrungsgemäß eher bedeckt, was Preise von
Fertigkomponten wie Signalen oder Ähnlichem betrifft. Nicht zuletzt werden in
der Realität die Preise für viele Komponten und Dienstleistungen nicht fix sein,
sondern für jeden Auftrag neu ausgehandelt.

Realistischer ist eine beobachtende Perspektive einzunehmen und vorhandene
Strecken bezüglich ihrer Kosten zu analysieren. Bei diesem Ansatz müssen vor
allem Daten über die Strecken erhoben werden. Im Idealfall würde man dazu
Einsicht in tatsächlich erfolgte Zahlungen und Detailpläne nehmen und dann
kompontenweise die Preise verschiedener Strecken vergleichen. Eine solche
Detailtiefe scheint aber zumindest bei Recherche über Fachmagazine und das
Internet nicht möglich. Sie ist auch für eine repräsentativ großem Menge an
Strecken sehr aufwendig. Da es sich im Rahmen unserer Möglichkeiten schon als
schwierig erwies, die Kosten für einen einzelnen Haltestelle zu ermitteln, haben
wir uns stattdessen darauf konzentriert, einge wichtige Daten für die Strecken
sowie den Gesamtpreis zu sammeln. Damit haben wir anschließend Kennzahlen wie
zum Beispiel Gesamtkosten pro Halt berechnet und diese dann in Relation
gesetzt. Diese Verfahren ist zwar praktikabel, jedoch stärker fehlerbehaftet. Eine
genaue Analyse der Fehlerquellen findet sich im vierten Abschnitt der Arbeit.

Um die verschiedenen Strecken innerhalb der Untersuchung vergleichen zu können,
müssen die zu ermittelnen Merkmale festgelegt werden. Dabei haben wir uns auf
folgende geeinigt, da sie sich meist in Fachzeitschriften,
Planfeststellungsverfahren oder auf den Websiten der Verkehrsbetriebe finden
lassen:

\newcounter{fnnumber}
\begin{itemize}
\item Streckenlänge
\item Tunnelanteile
\item Brückenanteil
\item Gleislänge \footnote{soweit vorhanden}\setcounter{fnnumber}{\thefootnote}
\item Weichenanzahl \footnotemark[\thefnnumber]
\item Anzahl der Haltepunkte und Bahnhöfe
\item tatsächliche bzw. geschätzte Gesamtkosten
\item Jahr der Fertigstellung
\end{itemize}

Wir vermuten, dass der wichtigste Faktor die Streckenlänge ist, wobei Brücken
bzw. Tunnelabschnitte teurer sind. Da Bahnhöfe und Haltepunkte zusätzliche
Ausstattungen und Gebäude erfordern, erscheinen uns diese auch als relevant für
die Gesamtkosten. Das Jahr der Fertigstellung wird benötigt, um die Inflation
beim Vergleich der Preise herauszurechnen.

Um die gennanten Daten zuerheben, haben wir vorallem Fachzeitschriften und
Artikel in Tageszeitungen benutzt. In manchen Fällen lieferten Pressemitteilungen
und Informationsbroschüren der jeweiligen Stadtverwaltungen die gesuchten Zahlen.
Bei einigen Verkehrsbetrieben haben wir auch direkt Anfragen gestellt, leider 
wurden nicht alle beantwortet. Die Ergebnisse unserer Recherche finden sich im 
nächsten Abschnitt.


\section{Ergebnisse}

\begin{comment}

Gliederung meiner Streckenbeschreibung (Yves)

* Übersicht
  * Umfeld des Bauprojekts
  * von wo nach wo
  * historische Einordnung

* konkrter Streckenverlauf
  * detalierte Daten (weichen etc ..)


* Probleme
  * bautechnisch
  * gesellschaftlich

* Zeitplan

* Kosten
\end{comment}



\subsection{Berlin}
\subsubsection*{Lückenschluss der U5}

Die U-Bahnlinie 5 fährt auf dem Abschnitt zwischen Alexanderplatz und
Friedrichsfelde schon seit 1930 \cite{bkhU5}. Schon im Vorfeld der
Eröffnung dieses Abschnittes, in den 20er Jahren, gab es Planungen, die Linie
Richtung Westen weiter zu verlängern \cite{bvgWebsiteU5}. Ein Vorgriff darauf
war die Eröffnung des Teilstücks U55 zwischen Brandenburger Tor (ehemals Unter
den Linden) und Hauptbahnhof. Die Schaffung dieser Insellösung wurde
nötig durch Verträge zwischen dem Land Berlin und dem Bund anlässlich der
Verlegung des Regierungssitzes \cite{hauptstadtvertrag}. Um die entstande Lücke
zu schließen, plant und baut die BVG eine Neubaustrecke zwischen Alexanderplatz
und Brandenburger Tor. Die ersten Bauarbeiten wurden 2010 begonnen, mit dem
Abschluss wird bisher 2019 gerechnet \cite{bvgWebsiteU5plan}. In unserer Arbeit
widmen wir uns auch der Untersuchung dieser Strecke.

Die neue Strecke beginnt am Alexanderplatz und verläuft vor dem Berliner Rathaus,
um dort auch schon die erste Station Berliner Rathaus zuerreichen
\cite{bkhU5} \cite{bvgWebsiteU5} Unmittelbar im Anschluß an den
Alexanderplatz wird ebenfalls eine Gleiswechselanlage errichtet. Im weiteren
(ebenfalls unterirdischen) Streckenverlauf wird die Spree von der Station
Museumsinsel unterquert. Ab dort verläuft die Strecke entlang der Straße Unter
den Linden, wo sie im gleichnamigen Kreuzungsbahnhof mit der Linie U6 hält und
anschließend an der Station Brandeburger Tor in die U55 endet. Die Gesamtläng der
Strecke beträgt damit etwa 2,2 km.

Der Bau wird komplett im Schildvortrieb ausgeführt, was durch den Sandboden vor
Ort erschwert wird \cite{bkhU5}. Desweiteren liegt die Strecke
mit bis zu 25 m vergleichsweise tief. Dies wird nötig, da sowohl die Spree als
auch die U6 unterquert werden müssen. Ersteres erfordert auch die Konstruktion
des Bahnhofs Museumsinsel im bergmännischen Verfahren, während die anderen
Bahnhöfe im Schachtverfahren gebaut werden können. Neben den baulichen
Schwierigkeiten ist auch der Nutzen des Projektes umstritten \cite{ftdU5}, da
die Strecke weitgehend parralell zu Stadtbahn verläuft und auch der
Kosten-Nutzenfaktor bis Turmstraße nur knapp über 1 liegt. Insgesamt werden die
Kosten für die Strecke auf 435 Mio. Euro geschätzt \cite{bwwwU5}. Wobei
sich zu dieser offiziellen Schätzung auch kritische Kommentare finden, die von
noch deutlich höhren Kosten ausgehen \cite{ftdU5}.

\subsubsection*{Straßenbahn zum Hauptbahnhof}

bla

\subsection{Hamburg - U4}

Die U4 in Hamburg ist eine 2012 neugeschaffene Linie, die sich allerdings die
meisten ihrer Haltstellen mit der U2 teilt. Ihre Aufgabe ist der Anchluss der
Hafencity an das Hamburger Hochbahnnetz \cite{keuHH}. Um dies
umzusetzen, wurde abzweigend von der vorhanden U2-Station Jungfernstieg eine neue
Strecke gebaut, die im Rahmen dieser Arbeit ebenfalls untersucht wird. Dabei
werden wir uns vor allem auf den schon fertigen Abschnitt konzentieren. Die
im Bau befindliche Verlängerung bleibt zunächst außen vor.

Die Neubaustrecke hat eine Gesamtlänge von ca. 4 km \cite{keuHH}. Sie
verläuft von Jungfernstieg ausgehend unter der U3 hindurch, um dann anschließend
das Hafenbecken zu unterqueren. Ihre beiden Haltestellen Überseequartier und
HafenCity Universität liegen dann beide auf dem Gebiet der Hafencity. Beide
Tunnelstationen konnten in offener Bausweise errichtet werden. Alle anderen
Streckenteile verlaufen ebenfalls unterirdisch und wurden im
Schildvortriebverfahren gebaut.

Neben den bautechnischen Schwierigkeiten der Hafenunterquerung hat auch dieses
Projekt mit Kritik auf verkehrspolitscher Ebene zu kämpfen. So wird die
unterirdische Streckenführung kritisiert, da sie teurer sei, als eine
Hochbahnlösung. \cite{hamburgerAbendblattu4}.

\subsection{Stuttgart}

Die Geschichte der Stuttgarter Straßenbahn geht zurück bis ins Jahr 1868. In den 1960er- und 1970er-Jahren wurden die damals oberirdisch liegenden Streckenabschnitte im Innenstadtbereich unter die Erde verlegt.
In den 1970er-Jahren gab es Planungen zum Aufbau eines U-Bahn-Netzes, es wurde sich jedoch dafür entschieden, das (meterspurige) Straßenbahnnetz sukzessive zur (normalspurigen) Stadtbahn umzubauen. Im Jahr 1985 wurde die erste Stadtbahnlinie am südlichen Stadtrand eröffnet, Ende 2007 wurde der Betrieb auf der Meterspur eingestellt \cite{SSBgeschichte}. Die im Folgenden vorgestellten Strecken der Linie U15 sind Teil dieses Umbaus.

\subsubsection*{U15 nach Ruhbank}



\subsubsection*{U15 nach Stammheim}

\subsubsection*{U12 nach Fasanenhof}


\subsection{Magdeburg}


\subsection{Nürnberg}


\subsection{Bremen}


\subsection{Freiburg}


\subsection{Darmstadt}


\subsection{Bergen}


\subsection{Århus}


\subsection{Toulouse}


\subsection{Los Angeles}


\subsection{Dubai}


\subsection{Brescia}






\begin{landscape}
\include{results_table}
\end{landscape}

%%% Local Variables:
%%% mode: latex
%%% TeX-master: "main"
%%% End:


\chapter{Auswertung}


% Tabelle mit durchschnitts Kosten und abweichung der bahnen davon
% Diagramm (Balken) mit Kosten pro km/station



\input{analysis_table_trams}

\hspace{1em}

\input{analysis_table_subway}


\chapter{Zusammenfassung}

\EUR{70} bis \EUR{80} U bahn 


\EUR{6 bis 13}


\bibliographystyle{gerplain}
\bibliography{main}

\end{document}
