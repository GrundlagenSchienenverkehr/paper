\chapter{Fazit}




Trotz der Schwierigkeit, genaue Daten zu finden, sind wir doch auf ein Ergebnis gekommen, mit dem wir zufrieden sind: Der preisliche Rahmen von U-Bahn-Strecken geht von \EUR{70 Mio} bis \EUR{75 Mio} pro Kilometer. Die Infrastrukturkosten von Straßenbahen liegen im Bereich \EUR{6 Mio} bis \EUR{13 Mio} pro Kilometer.

Aber diese Grenzen sind durchlässig. Bei U-Bahnen zeigt sich, wer optimale Bedingungen vorfindet, kann, wie Nürnberg mit der U3, diesen Bereich noch unterbieten. Prestigeobjekte, die der politischen Inszenierung dienen, können nach oben ausreißen, wie zum Beispiel der Lückenschluss zwischen Kanzler-U-Bahn und U5 in Berlin. 

Es zeigt sich, dass ein Kilometer U-Bahn 6 bis 10 mal so teuer ist wie ein Kilometer Straßenbahn. Anders ausgedrückt: Für einen Kilometer U-Bahn kann man etwa 8 Kilometer Straßenbahn bauen. Vor dem Bau von Nahverkehrsinfrastruktur sollte man sich die Frage stellen, was der Stadtbevölkerung mehr nützt. Die in manchen Städten von Bürgerinitiativen gebrachte Vorderung, statt einer U-Bahn eine Straßenbahn zu bauen\cite{fzNb}, kann durchaus Sinn machen.

Wir haben uns während der Erstellung dieser Arbeit die Frage gestellt, ob eine andere Herangehensweise nicht sinnvoller gewesen wäre. Statt einige wenige Quellen genau zu betrachten, hätten wir in die Breite gehen können. Dann hätten wir unsere Regressionen mit mehr Daten füttern könne und genauere Aussagen treffen können. Aber die Tendenz ist auch so klar gewurden und für uns war es spannender, uns tiefgreifend mit einzelnen Bahnen zu befassen, als im großem Stiel Daten zu sammeln und einzupflegen.  

Während unserer Recherchen musten wir feststellen, dass genaue und aufgeschlüsselte Infrastrukturkosten praktisch nicht zu finden waren. Noch mehr als die Verkehrsbetriebe und Stadtverwaltungen verbirgt die Industrie ihre Preise. Es scheint im beidseitigen Einverständnis zu liegen, das Bauverträge abseits der Öffentlichkeit ausgehandelt werden. Es bleibt zu hoffen, dass die aktuellen Open-Government-Bestrebungen, mittelfristig auch hier fruchtet und diese Daten zur Verfügung gestellt werden. 
