\chapter{Fazit}




Trotz der Schwierigkeit, genaue Daten zu finden, sind wir doch auf ein Ergebnis gekommen, mit dem wir zufrieden sind: Der preisliche Rahmen von U-Bahn-Strecken geht von \EUR{70 Mio} bis \EUR{75 Mio} pro Kilometer. Die Infrastrukturkosten von Straßenbahen liegen im Bereich \EUR{6 Mio} bis \EUR{13 Mio} pro Kilometer.

Aber diese Grenzen sind durchlässig. Bei U-Bahnen zeigt sich, wer optimale Bedingungen vorfindet, kann, wie Nürnberg mit der U3, diesen Bereich noch unterbieten. Prestigeobjekte, die der politischen Inszenierung dienen, können nach oben ausreißen, wie zum Beispiel der Lückenschluss zwischen Kanzler-U-Bahn und U5 in Berlin. 

Es zeigt sich, dass ein Kilometer U-Bahn 6 bis 10 mal so teuer ist wie ein Kilometer Straßenbahn. Anders ausgedrückt: Für einen Kilometer U-Bahn kann man etwa 8 Kilometer Straßenbahn bauen. Vor dem Bau von Nahverkehrsinfrastruktur sollte man sich die Frage stellen, was der Stadtbevölkerung mehr nützt. 

Während unserer Recherchen musten wir feststellen, dass genaue und aufgeschlüsselte Infrastrukturkosten praktisch nicht zu finden waren. Noch mehr als die Verkehrsbetriebe und Stadtverwaltungen verbirgt die Industrie ihre Preise. Es scheint im beidseitigen Einverständnis zu liegen, das Bauverträge abseits der Öffentlichkeit ausgehandelt werden. Unsere persönliche Hoffnung ist, dass die Open-Government-Bestrebungen in der deutschen Demokratie mittelfristig auch hier fruchtet. 
